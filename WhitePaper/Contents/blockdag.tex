The blockdag, or a DAG of \emph{blocks}, is the key structure for the
protocol.

\subsection{Block}
A \textbf{block} is a public key certificate that consists of:
\begin{itemize}
    \item a \emph{header}
    \begin{itemize}
        \item{identifier of the creator of the block}
        \item{list of hashes of the parent blocks}
    \end{itemize}
    \item a \emph{body}
        \begin{itemize}
            \item{list of \emph{transactions}}
                \begin{itemize}
                    A \emph{transaction} is a pair. (\textbf{topic, event})
                \end{itemize}
        \end{itemize}
    \item signature of the creator
        \begin{itemize}
            \item{The signed items include the header and body}
        \end{itemize}
\end{itemize}
A block is uniquely identified within a structure by the hash over all of its
contents. Within the structure, there exists one unique block that has no
parent hashes. The name of this block is the \textbf{genesis block}.

\subsection{Blockdag}
The blockdag structure can be defined as a DAG where the vertices are
the blocks. The edges are originating at vertex and point towards the vertex
described by the vertex's parent hash. The structure has the following
constraints:
\begin{enumerate}
    \item{There exists a unique genesis block AND it is the \emph{sole} sink vertex}
    \item{There exists a path from every block to the genesis block}
    \item{There exists \textbf{no} edge $b_1 \rightarrow b_2$ if there is 
    another path $b_1$ to $b_2$}
\end{enumerate}
The \textbf{frontier set} is the set of blocks on the DAG that have no
successors (sources, given that blocks in the DAG point to their parent
blocks).

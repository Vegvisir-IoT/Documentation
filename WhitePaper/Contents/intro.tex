\chapter*{Introduction}
\chaptermark{Introduction}
\addcontentsline{toc}{chapter}{Introduction}

As presented in its introductory paper\cite{karlsson2018vegvisir}, Vegvisir is
a tamperproof log that is partition-tolerant. Tamperproof logs, often defined
as a blockchains, often have large power constraints that is associated with
its security protocol, proof of work. This protocol allows for any entity to
join the blockchain. Vegvisir differs from other implementations, because it is
has a permissioned membership protocol and only supports Conflict-free
replicated data types (CRDTs). The DAG structure itself creates a partial
ordering, and the use of CRDTs means that transactions are communtative. These
key differences contribute for its ability to be work on IoT devices and in
environments with unstable network connectivity.

In this paper, we will present the fundamental concepts required to being able
to recreate Vegvisir. The information will be presented in the following order:
\begin{enumerate}
    \item General Overview
    \item Vegvisir Core
    \item Hardware Interface
    \item Application Interface
\end{enumerate}

As this paper is dedicated to the building of the protocol, some concepts be
given a basic cursurary explanation. The explainations are enough to build the
protocol, yet they are not complete in all cases. Therefore, citations will be
included in the text and located in Section \ref{reference}.
